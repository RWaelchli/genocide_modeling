\documentclass[11pt]{article}
\usepackage{geometry}                
\geometry{letterpaper}                   

\usepackage{graphicx}
\usepackage{amssymb}
\usepackage{epstopdf}
\usepackage{natbib}
\usepackage{amssymb, amsmath}
\DeclareGraphicsRule{.tif}{png}{.png}{`convert #1 `dirname #1`/`basename #1 .tif`.png}

%\title{Title}
%\author{Name 1, Name 2}
%\date{date} 

\begin{document}



\thispagestyle{empty}

\begin{center}
\includegraphics[width=5cm]{ETHlogo.eps}

\bigskip


\bigskip


\bigskip


\LARGE{ 	Lecture with Computer Exercises:\\ }
\LARGE{ Modelling and Simulating Social Systems with MATLAB\\}

\bigskip

\bigskip

\small{Project Report}\\

\bigskip

\bigskip

\bigskip

\bigskip


\begin{tabular}{|c|}
\hline
\\
\textbf{\LARGE{Civil Violence: Modelling Genocides;}}\\
\textbf{\LARGE{Model Extension and Global Sensitivity Analysis}}\\
\\
\hline
\end{tabular}
\bigskip

\bigskip

\bigskip

\LARGE{Ruben W\"alchli \& Zoran Bjelobrk}



\bigskip

\bigskip

\bigskip

\bigskip

\bigskip

\bigskip

\bigskip

\bigskip

Zurich\\
\today\\

\end{center}



\newpage

%%%%%%%%%%%%%%%%%%%%%%%%%%%%%%%%%%%%%%%%%%%%%%%%%

\newpage
\section*{Agreement for free-download}
\bigskip


\bigskip


\large We hereby agree to make our source code for this project freely available for download from the web pages of the SOMS chair. Furthermore, we assure that all source code is written by ourselves and is not violating any copyright restrictions.

\begin{center}

\bigskip


\bigskip


\begin{tabular}{@{}p{3.3cm}@{}p{6cm}@{}@{}p{6cm}@{}}
\begin{minipage}{3cm}

\end{minipage}
&
\begin{minipage}{6cm}
\vspace{2mm} \large Ruben W\"alchli

 \vspace{\baselineskip}

\end{minipage}
&
\begin{minipage}{6cm}

\large Zoran Bjelobrk

\end{minipage}
\end{tabular}


\end{center}
\newpage

%%%%%%%%%%%%%%%%%%%%%%%%%%%%%%%%%%%%%%%



% IMPORTANT
% you MUST include the ETH declaration of originality here; it is available for download on the course website or at http://www.ethz.ch/faculty/exams/plagiarism/index_EN; it can be printed as pdf and should be filled out in handwriting


%%%%%%%%%% Table of content %%%%%%%%%%%%%%%%%

\tableofcontents

\newpage

%%%%%%%%%%%%%%%%%%%%%%%%%%%%%%%%%%%%%%%



\section{Abstract}

\section{Individual contributions}

\section{Introduction and Motivations}
Ethnical conflicts were omnipresent throughout human history and still are a major source for civil violence in these days. For instance, the wars leading to the decay of former Yugoslavia are an example in recent history. When studying the project suggestions, this topic immediately drew our interest. Our project is a so-called "model-driven" project. The base model was described by Epstein in 2002. It contains some very strong simplifications as further described in section \ref{subsec:base_model}. Our approach was to modify the model in a way that describes reality better in our opinion and to compare the results of the model runs to those presented in literature. To further assess the model in a quantitative manner, a global sensitivity analysis was performed for various model parameters based on a method described in literature.

\section{Description of the Model}
\label{sec:description_model}

\subsection{Base Model}
\label{subsec:base_model}
The model is a so-called agent-based model. There are two fundamental classes of agents: civilians and law enforcement officers (LEOs). The only purpose of the LEO is to prevent violence through arresting active, this means violent, civilians. The civilians are further divided into two ethnical groups. They become active or stay quiet (peaceful) based on the values of their individual characteristics and the environment within their range of sight. The agents are placed on a two-dimensional map on which they can move and interact with each other.

\subsubsection{Specification of the Civilians}
The will of a civilian to become violent is described by its grievance $G$. This individual grievance is described by two components: the hardship $H$ of the civilian and his/her perceived legitimacy of the other ethnic group $L$.
\begin{align}
G = H (1 - L)
\end{align}
The grievance of the individual civilian is according to this formula given by the hardship multiplied by the "illegitimacy" $(1 - L)$ of the other ethnic group. The hardship is heterogeneous across the civilians. It is initialized for each civilian at the beginning of the simulation by drawing it from the uniform distribution on the interval $[0,1]$ ($U(0,1)$) and it remains constant over the course of the simulation. $L$ is homogeneous across all civilians and also stays constant. It is also equal for both ethnic groups. It is defined at the beginning of the simulation and also lies between 0 and 1.\\
\\
$R$ is the risk aversion of the civilian. It is heterogeneous across the population and also drawn upon initialization from $U(0,1)$. A civilian inspects its environment before deciding whether to go active or stay quiet (and vice versa). The civilians have the vision $v$, which is homogeneous across the population and is defined at the beginning of the simulation.
\begin{align}
P = 1 - \exp \left( - k \left( \frac{LEO}{A} \right)_v \right)
\end{align}
$P$ is the arrest probability estimated by the civilian. $\left(LEO/A\right)_v$ is the LEO-to-active ratio within the civilians vision. $A$ is always at least $1$, because the civilian counts himself as active when calculating the arrest probability. The reasoning behind this calculation is, that it is less likely for the civilian to be arrested, when already a lot of other civilians are acting violently and very few LEOs are present around him. The opposite can be said, if there are a lot of LEOs and only few or no active civilians. $k$ is a constant parameter, that is calculated by setting $P = 0.9$ when $\left(LEO/A\right)_v = 1$. With the risk aversion and the estimated arrest probability the so-called net risk of the civilian of going active can be calculated:
\begin{align}
N = RP
\end{align}
The difference $G - N$ is the civilian's net utility of becoming active. This is compared to a threshold value $T$. If $(G - N) > T$ the civilian becomes or stays active and if $(G - N) \leq T$ the civilian stays or becomes quiet. The threshold value $T$ is equal for all civilians and defined at the beginning of the simulation.

\subsubsection{Specification of the LEOs}
The LEOs have only one characteristic, their vision $v^\star$. They inspect the map within their vision and randomly arrest an active civilian.

\subsubsection{Evolution of the Simulation}
After choosing values for the global parameters, the three different types of agents are placed on the map randomly and their individual attributes are initialized. The number of civilians and LEOs that are placed on the map are determined by their corresponding selected densities. In each iteration begins by the random selection of one agent (civilian or LEO). This agent moves to a random empty position within his vision. There he inspects his environment. If the agent is a civilian he takes the decision to be active or quiet. When he decides to become active, he tries to kill a random civilian of the other ethnic group within his vision. If the selected agent is a LEO, he arrests a random active civilian within his vision. The arrested civilian is assigned a jail term and moved to the jail. The jail term is drawn from $U(0,J_{\text{max}})$. The ages both of the civilians on the map as well as of those in jail are updated and the civilians that have reached their maximum age are removed. The number of turns of the inmates spent in jail are updated and those who have served their time are released onto random positions on the map.

\subsection{Modification of the Model}

\subsubsection{Individual Perceived Legitimacy and Threshold}
In our opinion, the assumption that the perceived legitimacy of the other ethnic group and the threshold value are identical for all individuals of the population is too harsh. Therefore, we wanted to make them individual quantities. In the modified model the perceived legitimacy of the other ethnic groups is also initialized for all civilians by drawing it from $U(0,1)$.\\
\\
The threshold $T$ can be seen as the level of "extremism" an individual has. Epstein describes it as the utility of not publicly expressing one's private grievance. A very extreme individual will see no utility in holding his private grievance back, whether a very conservative individual will almost never publicly express his private grievance. The difference $(G-N)$ can take values in the interval $[-1,1]$. Therefore it is sensible to draw the individual threshold values from the uniform distribution defined in that range. A civilian with $T = -1$ will always be active, whereas a civilian with $T = 1$ will never be active.

\subsubsection{Updating the Perceived Legitimacy}
It seems quite obvious that the perceived legitimacy of the other ethnic group should not be a constant over the course of the simulation. Violence committed by one ethnic group will probably decrease its perceived legitimacy in the eyes of the other ethnic group. The opposite effect will possibly have the arrest of a active (violent) civilian on the perceived legitimacy of its ethnic group. In our expansion of the model, these effects are limited to the range of sight of the involved active civilian. The update in the case of violence is the following:
\begin{align}
L_{\text{new}} = L_{\text{old}} \left( 1 - k_L \right)
\end{align}
When an active civilian is arrested, the updates is given by:
\begin{align}
L_{\text{new}} = L_{\text{old}} + k_L \left( 1 - L_{\text{old}} \right)
\end{align}
The reasoning behind both formulas is that if the perceived legitimacy is either very high or very low, violence and arrests only change little (saturation effect). On the other hand, when a civilian is unsure, if the other ethnic group has the right to exist (medium value of $L$) such events can have a strong influence.

\section{Implementation}

\subsection{Base Model}
The whole implementation is based on arrays of structs of which the one describing the map is in the center of the program. This way of implementation may not be the fastest possible, but in our opinion the most intuitive one.

\subsection{Sensitivity Analysis}
To evaluate the sensitivity of the model parameters numerically, the sensitivity analysis described by Saltelli (cite Saltelli2002) was implemented. The cumulative number of killings after 200 time steps was chosen as reasonable output variable $y = f(x_1,x_2,...,x_k)$ to quantify the sensitivity of each of the input parameters $x_i$.
The variables tested for their sensitivity on the output variable $y$ are summarized in table \ref{table:SensitivityVariables}. 

\begin{table}[h]
\begin{tabular}{|l|cccc|}
\hline
 &  &  &  &  \\ \hline
 &  &  &  &  \\ \hline
 &  &  &  &  \\ \hline
 &  &  &  &  \\ \hline
 &  &  &  &  \\ \hline
 &  &  &  &  \\ \hline
 &  &  &  &  \\ \hline
 &  &  &  &  \\ \hline
 &  &  &  &  \\ \hline
 &  &  &  &  \\ \hline
 &  &  &  &  \\ \hline
\end{tabular}
\end{table}


According to Saltelli, a Monte Carlo sampling of the output variable y was performed by generating random variables with the Sobol method of MATLAB





\subsubsection{Structure Describing the Agents}


\section{Simulation Results and Discussion}





\section{Summary and Outlook}

\section{References}






\end{document}  



 
